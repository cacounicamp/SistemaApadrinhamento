% Documento do tipo report (tese, dissertações, relatórios) em A4
\documentclass[10pt]{article}


% Permite a escrita com caracteres UTF-8
\usepackage[utf8]{inputenc}
% Documento com "Capítulo", "Seção" escrito em português
\usepackage[brazil]{babel}
% Indentar o primeiro parágrafo após seções
\usepackage{indentfirst}
% Adicionamos enumerações
\usepackage{enumitem}
% Adicionamos links
\usepackage{hyperref}


% Título e autores do documento, sem mostrar data
\title{Manual de Apadrinhamento}
% Os autores foram escritos manualmente, pois não cabia na página. Pular linha
% não funcionou
\author{Centro Acadêmico da Computação -- CACo}
\date{\the\year}


% Iniciamos o documento
\begin{document}

% Incluímos o título
\maketitle

% Incluímos o manual ;)
\section*{O que é o sistema de apadrinhamento?}

O sistema de apadrinhamento é um programa do CACo que procura fazer com que
todas as bixetes e bixos dos cursos ciência e engenharia de computação tenham,
respectivamente, uma veterana como madrinha ou um veterano como padrinho, que
ajudem na transição entre o ensino médio e a faculdade.

\section*{Por que existe o programa?}

A grande parte das bixetes e dos bixos que chegam conseguem se enturmar com
seus colegas, veteranas e veteranos logo nas atividades de integração ou já
possuem algum conhecido na Unicamp. No entanto, isso não acontece com todas as
pessoas: é possível que tenham dificuldades que poderiam ser evitadas com a
ajuda de alguma veterana ou veterano que pudesse facilitar introdução e
familiarização à vida na universidade.

\section*{Como faço para ser uma madrinha ou um padrinho?}

Junto com esse manual, haverá um formulário de cadastro simples para as
veteranas candidatas a madrinha e veteranos candidatos a padrinho responderem.
Após respondê-lo, seu nome será adicionado à lista de madrinhas ou padrinhos.

A partir da primeira matrícula presencial (19 de fevereiro), será sorteada uma
bixete para ser sua afilhada ou um bixo para ser seu afilhado. Caso não haja um
número de madrinhas e padrinhos igual ao número de bixetes e bixos, poderá
receber mais de uma afilhada ou afilhado, de acordo com suas respostas no
questionário.

\section*{O que tenho que fazer como madrinha ou padrinho?}

Como madrinha ou padrinho, seu interesse é que a sua afilhada ou seu afilhado
tenha um ingresso na faculdade divertido e sem traumas, que tenha consciência
sobre as oportunidades da Unicamp e, por fim, possa programar melhor o seu
futuro. Para que a experiência dê certo, é necessário que procure estabelecer
um relacionamento pessoal e sincero com a afilhada ou o afilhado, de modo a
ganhar sua confiança e poder entender seus problemas. Lembre-se de como foram
seus primeiros dias na Unicamp.

Sabemos que nem todas as bixetes e bixos são iguais, alguns sendo bastante
tímidos. Por isso é essencial que o padrinho apresente uma posição ativa nesse
relacionamento, assumindo a iniciativa sempre que possível. Para te ajudar, o
CACo preparou duas listas: uma de informações que devem ser passadas logo na
primeira semana sem que a bixete ou o bixo precise perguntar alguma coisa, e
uma de sugestões que pode adotar para manter contato com sua afilhada ou seu
afilhado e manter um relacionamento saudável. Essa lista é apenas de sugestões,
e você pode tentar outras formas de interação com a afilhada ou afilhado,
sempre respeitando as vontades e limites.

\subsection*{Informações básicas que devem ser passadas na primeira semana:}
\begin{itemize}[noitemsep] % 'noitemsep' = menos espaço entre os itens
\item Horários e funcionamento dos bandejões;
\item Moradia (Moradia da Unicamp, Repúblicas Unicamp, Morar Unicamp etc);
\item Redirecionamento de emails da DAC, do IC e da FEEC;
\item Dúvidas sobre o sistema da DAC;
\item GDE;
\item Manual d* Bix* (existe a \href{http://www.caco.ic.unicamp.br/manual.pdf}
  {versão PDF}).
\end{itemize}

\subsection*{Sugestões para interação continuada com a afilhada ou afilhado:}
\begin{itemize}[noitemsep]
\item Marcar um dia da semana para bandejarem juntos;
\item Convidar a afilhada ou afilhado para os programas que costuma fazer,
  como ir para algum bar, visitar amigas e/ou amigos, jogar videogame ou jogos
  de tabuleiro, jantar fora etc;
\item Apresentar suas amigas e seus amigos, colegas de turma, pessoas
  conhecidas da Unicamp;
\item Fazer amizades com colegas da afilhada ou afilhado, saírem juntos;
\item Manter uma conversa contínua por redes sociais ou e-mail.
\end{itemize}

\section*{Por que eu deveria ser madrinha ou padrinho?}

A resposta mais honesta para isso é que acreditamos que um sistema de
apadrinhamento seria benéfico para toda a comunidade da computação da Unicamp,
criaria um ambiente em que ingressantes tenham mais interação com suas
veteranas e veteranos, mais consciência das oportunidades oferecidas aqui. Além
disso, acreditamos que, ao se tornar uma madrinha ou padrinho, ganhará novas
amizades, desenvolverá suas habilidades de comunicação, muito importante para
a sua vida pessoal e profissional.

\section*{Sou madrinha/padrinho e estou tendo problemas. O que faço?}

Algum membro do CACo ficará responsável por ouvir suas impressões e buscar
soluções imediatas, além de propor melhorias para o programa no futuro. Você
deverá procurar o centro acadêmico quando a situação for irrecuperável. Se for
necessário, alocaremos sua afilhada ou afilhado para uma nova madrinha ou
padrinho. Poderá nos encontrar no
\href{https://www.facebook.com/cacounicamp/}{Facebook} ou nos mandando um
e-mail em \href{mailto:caco@ic.unicamp.br}{caco@ic.unicamp.br}

Isso pode acontecer por alguns motivos: você pode ter subestimado as atividades
que teria para fazer durante o semestre e não tem mais condições de ser uma boa
madrinha ou bom padrinho; pode ter surgido algum imprevisto na vida acadêmica
ou pessoal sua ou na de sua afilhada ou afilhado ou ainda houve uma enorme
incompatibilidade entre você e sua afilhada ou afilhado, de modo que não há
mais como manter a relação.

O importante é dar todo o conhecimento necessário à afilhada ou ao afilhado,
mesmo que não haja grande compatibilidade nos interesses.

\section*{Já tenho uma afilhada ou afilhado, mas gostei de outra(o) ingressante
também. Posso pedir para ser minha afilhada ou afilhado?}

O objetivo do sistema é criar esse ambiente de amizades e conexões entre várias
turmas da computação. Vá em frente, explique a Unicamp a quem você puder, faça
a integração que você gostaria de receber. Participe de rolês e conheça
bixetes, bixos, veteranas e veteranos que nunca tinha visto.\\

Agradecemos a disposição, esperamos que participe!

\end{document}
